$-- You can use as many custom partials as you need. Convention is to prefix name with '_'
$-- It can be useful to use such template to split some template parts in smaller pieces, which is easier to reuse. 
$-- This '_custom.tex' is used on 'title.tex' as example.
$-- See other existing format in quarto-journals/ organisation.
$-- %%%% TODO %%%%%
$-- Use it if you need to insert content at this specific place of the main Pandoc's template. Otherwise, remove it.
$-- Here we are using it to format the authors part of the template.
$-- %%%%%%%%%%%%%%%



%%%%%%%%%%%%%%%%%%%%%%%%%%%%%%%%%%%%%%%%%%%%
%% partials/_authors.tex
%%%%%%%%%%%%%%%%%%%%%%%%%%%%%%%%%%%%%%%%%%%%

% Author information does not appear in the pdf unless the "acceptedWithA" option is given

% The author block may be formatted in one of two ways:

$if(tacl-options)$
$if(tacl-options.styleone)$

% Option 1. Author’s address is underneath each name, centered.

\author{
  $for(by-author)$$it.name.literal$ \\
    $if(by-author.affiliations/first)$$for(by-author.affiliations/first)$ $_print-affiliation.tex()$$endfor$$endif$
    $if(it.email)$\texttt{$it.email$} \\ $endif$
    $sep$\And $endfor$\ \\
}

$else$

% Option 2.  Author’s address is linked with superscript
% characters to its name, author names are grouped, centered.

\author{
$for(by-author)$$it.name.literal$ $$^$it.metadata.symbol$$$ $if(it.symbol)$$it.symbol.literal$$endif$ $sep$\and $endfor$ \\
  \\ $for(by-author)$ $$^$it.metadata.symbol$ $$  $if(by-author.affiliations/first)$$for(by-author.affiliations/first)$ $_print-affiliation.tex()$$endfor$$endif$
$if(it.email)$\texttt{$it.email$} \\ 
\ \\
$endif$$endfor$\ \\
}

$endif$
$endif$

%%%%%%%%%%%%%%%%%%%%%%%%%%%%%%%%%%%%%%%%%%%%
%% end partials/_authors.tex
%%%%%%%%%%%%%%%%%%%%%%%%%%%%%%%%%%%%%%%%%%%%
